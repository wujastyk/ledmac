%%
%% This is file `ledmixed.tex',
%% generated with the docstrip utility.
%%
%% The original source files were:
%%
%% ledmac.dtx  (with options: `periph')
%% 
%%   Author: Peter Wilson (Herries Press) herries dot press at earthlink dot net
%%   Copyright 2003 -- 2005 Peter R. Wilson
%% 
%%   This work may be distributed and/or modified under the
%%   conditions of the LaTeX Project Public License, either
%%   version 1.3 of this license or (at your option) any
%%   later version.
%%   The latest version of the license is in
%%      http://www.latex-project.org/lppl.txt
%%   and version 1.3 or later is part of all distributions of
%%   LaTeX version 2003/06/01 or later.
%% 
%%   This work has the LPPL maintenance status "unmaintained".
%% 
%%   This work consists of the files listed in the README file.
%% 
\documentclass{article}
\usepackage{ledmac}

 \noendnotes
%% \overfullrule0 pt
 \lefthyphenmin=3

\let\Ma=$
\let\aM=$
\usepackage[delims]{lgreek}

 % We need an addition to \no@expands since the \active $ in lgreek
 % causes problems:
 \newcommand{\morenoexpands}{\let$=0}

\makeatletter

 \newbox\lp@rbox

 \newcommand{\ffootnote}[1]{%
   \ifnumberedpar@
     \xright@appenditem{\noexpand\vffootnote{f}{{\l@d@nums}{\@tag}{#1}}}%
                                                 \to\inserts@list
     \global\advance\insert@count by 1
 %  \else         %% may be used only in numbered text
 %    \vffootnote{f}{{0|0|0|0|0|0|0}{}{#1}}%
   \fi\ignorespaces}

 \newcommand{\gfootnote}[1]{%
   \ifnumberedpar@
     \xright@appenditem{\noexpand\vgfootnote{g}{#1}}%
                                                 \to\inserts@list
     \global\advance\insert@count by 1
 %  \else         %% may be used only in numbered text
 %    \vgfootnote{g}{#1}%
   \fi\ignorespaces}

 \newcommand{\setlp@rbox}[3]{%
   {\parindent\z@\hsize=2.5cm\raggedleft\scriptsize
   \baselineskip 9pt%
   \global\setbox\lp@rbox=\vbox to\z@{\vss#3}}}

 \newcommand{\vffootnote}[2]{\setlp@rbox#2}

 \newcommand{\vgfootnote}[2]{\def\rd@ta{#2}}

 \renewcommand{\affixline@num}{%
   \ifsublines@
     \@l@dtempcntb=\subline@num
     \ifnum\subline@num>\c@firstsublinenum
       \@l@dtempcnta=\subline@num
       \advance\@l@dtempcnta by-\c@firstsublinenum
       \divide\@l@dtempcnta by\c@sublinenumincrement
       \multiply\@l@dtempcnta by\c@sublinenumincrement
       \advance\@l@dtempcnta by\c@firstsublinenum
     \else
       \@l@dtempcnta=\c@firstsublinenum
     \fi
     %
     \ifcase\sub@lock
       \or
         \ifnum\sublock@disp=1
            \@l@dtempcntb=0 \@l@dtempcnta=1
         \fi
       \or
         \ifnum\sublock@disp=2 \else
            \@l@dtempcntb=0 \@l@dtempcnta=1
         \fi
       \or
         \ifnum\sublock@disp=0
            \@l@dtempcntb=0 \@l@dtempcnta=1
         \fi
     \fi
   \else
     \@l@dtempcntb=\line@num
     \ifnum\line@num>\c@firstlinenum
        \@l@dtempcnta=\line@num
        \advance\@l@dtempcnta by-\c@firstlinenum
        \divide\@l@dtempcnta by\c@linenumincrement
        \multiply\@l@dtempcnta by\c@linenumincrement
        \advance\@l@dtempcnta by\c@firstlinenum
     \else
        \@l@dtempcnta=\c@firstlinenum
     \fi
     \ifcase\@lock
        \or
          \ifnum\lock@disp=1
             \@l@dtempcntb=0 \@l@dtempcnta=1
          \fi
        \or
          \ifnum\lock@disp=2 \else
             \@l@dtempcntb=0 \@l@dtempcnta=1
          \fi
        \or
          \ifnum\lock@disp=0
             \@l@dtempcntb=0 \@l@dtempcnta=1
          \fi
     \fi
   \fi
   %
   \ifnum\@l@dtempcnta=\@l@dtempcntb
     \@l@dtempcntb=\line@margin
     \ifnum\@l@dtempcntb>1
       \advance\@l@dtempcntb by\page@num
     \fi
     \ifodd\@l@dtempcntb
 %      #1\rlap{{\rightlinenum}}%
        \xdef\rd@ta{\the\line@num}%
     \else
       \llap{{\leftlinenum}}%#1%
     \fi
   \else
     %#1%
   \fi
   \ifcase\@lock
   \or
     \global\@lock=2
   \or \or
     \global\@lock=0
   \fi
   \ifcase\sub@lock
   \or
     \global\sub@lock=2
   \or \or
     \global\sub@lock=0
   \fi}

 \lineation{page}
 \linenummargin{right}
 \footparagraph{A}
 \footparagraph{B}

\renewcommand{\notenumfont}{\footnotesize}
\newcommand{\notetextfont}{\footnotesize}

 \let\Afootnoterule=\relax
 \count\Afootins=825
 \count\Bfootins=825

 \newcommand{\Aparafootfmt}[3]{%
   \ledsetnormalparstuff
   \scriptsize
   \notenumfont\printlines#1|\enspace
 %      \lemmafont#1|#2\enskip
   \notetextfont
   #3\penalty-10\hskip 1em plus 4em minus.4em\relax}

 \newcommand{\Bparafootfmt}[3]{%
   \ledsetnormalparstuff
   \scriptsize
   \notenumfont\printlines#1|\enspace
   \select@lemmafont#1|#2\rbracket\enskip
   \notetextfont
   #3\penalty-10\hskip 1em plus 4em minus.4em\relax }
 \makeatother

 \let\Afootfmt=\Aparafootfmt
 \let\Bfootfmt=\Bparafootfmt
 \def\lemmafont#1|#2|#3|#4|#5|#6|#7|{\scriptsize}
 \parindent=1em

 \newcommand{\lmarpar}[1]{\edtext{}{\ffootnote{#1}}}
 \newcommand{\rmarpar}[1]{\edtext{}{\gfootnote{#1}}}
 \emergencystretch40pt

%%%%%%%%%%%%%%%%%%%%%%%%%%%%%%%%%%%%%%%%%%%%%%

\begin{document}

 \beginnumbering
 \pstart
 \rmarpar{741C}
 \noindent \edtext{Incipit Quartus $PERIFUSEWN$}{%
 \lemma{incipit\ .~.~.\ $PERIFUSEWN$}\Bfootnote{\textit{om.\ R},
 incipit quartus \textit{M}}}
 \pend
 \medskip

 \pstart
 \noindent \edtext{NVTRITOR}{\lemma{$ANAKEFALIOSIS$}\Bfootnote{\textit{
 FJP, lege} $<anakefala'iwsis$}}.\lmarpar{$ANAKEFALIOSIS$
 NATVRARVM} Prima nostrae
 \edtext{Physiologiae}{\lemma{physiologiae}\Bfootnote{phisiologiae
 \textit{P}, physeologiae \textit{R}}}
 intentio praecipuaque mat\-e\-ria erat
 \edtext{quod}{\Bfootnote{\textit{p}.\ natura \textit{transp.\ MR}}}
 \edtext{$UPEROUSIADES$}{\Bfootnote{\textit{codd.\ Vtrum}
 $<uperousi'wdhs$ (hoc est superessentialis) natura \textit{cum Gale
 (p.160) an} $<uperousi'oths$ (hoc est superessentialis natura)
 \textit{cum Floss (PL 122,741C) intelligendum sit, ambigitur}}}
 (hoc est superessentialis) natura sit causa creatrix existentium et
 non existentium omnium, a nullo creata, unum principium, una
 origo, unus et uniuersalis uniuersorum fons, a nullo manans, dum
 ab eo manant omnia, trinitas coessentialis in tribus substantiis,
 $ANARQOS$ (hoc est sine principio), principium et finis, una
 bonitas, deus unus,
 \edtext{$OMOUSIOS$}{\Bfootnote{\textit{codd., lege} $<omoo'usios$}}
 \edtext{et}{\lemma{\textbf{et}}\Bfootnote{\textit{
 R}\textsuperscript{1}, \textit{om.\ R}\textsuperscript{0}}}
 $UPEROUSIOS$ (id est coessentialis et superessentialis). Et, ut
 ait sanctus Epifanius, episcopus Constantiae Cypri, in
 \edtext{$AGKURATW$}{\Bfootnote{anchurato \textit{MR}}}
 sermone
 \edtext{de fide}{\Bfootnote{Glo\Ma\langle\aM ssa\Ma\rangle\aM: Ita
 enim uocatur sermo eius de fide $AGKURATOS$, id est procuratus
 \textit{mg.\ add.\ FJP}}}:
 \begin{itshape}Tria sancta, tria consancta, tria
 \edtext{agentia}{\Bfootnote{actiua \textit{MR}}},
 tria coagentia, tria
 \edtext{formantia}{\Bfootnote{formatiua \textit{MR}}},
 tria conformantia, tria
\edtext{operantia}{\Bfootnote{operatiua \textit{MR}}},
 tria cooperantia, tria subsistentia, tria\rmarpar{742C}
 consubsistentia sibi inuicem coexistentia. Trinitas haec
 sancta uocatur: tria existentia, una consonantia, una deitas
 \edtext{eiusdem}{\Bfootnote{eiusdemque \textit{M}}}
 essentiae,
 \edtext{eiusdem uirtutis, eiusdem
   \edtext{subsistentiae}{\Bfootnote{substantiae \textit{R}}}}{%
 \Bfootnote{\textit{om.\ M}}},
 similia
\edtext{similiter}{\Bfootnote{ex simili \textit{MR}}}
 aequalitatem gratiae operantur patris et filii et sancti spiritus.
 Quomodo autem
 \edtext{sunt}{\Bfootnote{\textit{om.\ M}}},
 ipsis relinquitur docere:
 \edtext{`Nemo enim nouit patrem nisi filius, neque filium nisi pater,
    et cuicumque filius reuelauerit'}{\Afootnote{Matth.\ 11, 27}};
 reuelatur autem per spiritum sanctum. Non ergo haec tria existentia
 aut ex ipso aut per ipsum aut ad ipsum in unoquoque digne intelliguntur,
 \Ma\mid\! R, 264^{\rm r}\!\mid\aM\ sicut ipsa reuelant:\end{itshape}
 $FWS, PUR, PNEUMA$
 \edtext{(hoc est lux, ignis, spiritus)}{\Afootnote{EPIPHANIVS,
  \textit{Ancoratus} 67; PG~43, 137C--140A; GCS 25, p.~82, 2--12}}.
 \pend

 \pstart
 Haec, ut dixi, ab Epifanio tradita, ut quisquis interrogatus quae
 tria et quid unum in sancta trinitate debeat credere, sana fide
 \Ma\!\mid J, 1^{\rm v}\!\mid\aM\ respondere ualeat, aut ad
 fidem accedens\rmarpar{743A} sic erudiatur. Et mihi uidetur
 spiritum pro calore posuisse, quasi dixisset in similitudine:
 lux, ignis, calor. Haec enim tria unius essentiae sunt. Sed cur
 lucem primo dixit, non est mirum. Nam et pater lux est et ignis
 et calor; et filius est lux, ignis, calor; et
 \edtext{spiritus sanctus}{\Bfootnote{sanctus spiritus \textit{R}}}
 lux, ignis, calor. Illuminat enim pater, illuminat filius, illuminat
 spiritus sanctus: ex ipsis enim omnis scientia et sapientia donatur.
 \pend
 \endnumbering

\end{document}

\endinput
%%
%% End of file `ledmixed.tex'.
